\section{Introduction}

\begin{itemize}
  \item
    State machine replication is a very important thing.

  \item
    With traditional state machine replication, every command has to be
    executed by every replica, so the protocol cannot run faster than a single
    replica.

  \item
    This is safe but overly restrictive. Recent state machine replication
    protocols have noted that while writes have to be executed by all replicas,
    reads can be processed by a single replica. So, we can scale up the number
    of replicas and scale up the read throughput of the system. This is
    especially useful since a lot of workloads are read-heavy.

  \item
    Existing protocols, however, have limitations. Harmonia requires
    specialized hardware and assumes clock synchrony for correctness. CRAQ is
    sensitive to data skew and is forced to trade off read throughput and write
    latency. It's also based on Chain Replication, so the failure of any
    machine stalls the protocol.

  \item
    This paper presents two main contributions. First, we present a novel state
    machine replication protocol with scalable read throughput that doesn't
    suffer the limitations of existing approaches. Our protocol, called Evelyn
    Paxos, combines two existing protocols, Compartmentalized MultiPaxos and
    Linearizable Quorum Reads, in a straightforward way.

  \item
    Second, we prove a surprising impossibility result. Existing papers claim
    to achieve ``scalable throughput'', but what exactly does this mean? Well,
    imagine we have a workload with a fixed write fraction. Given a desired
    throughput, we would like to be able to scale up the number of replicas to
    achieve this throughput. Surprisingly, this is impossible. We prove that
    for any state machine replication protocol in which writes are executed by
    all replicas (e.g., Evelyn Paxos, Harmonia, CRAQ), and for any workload
    with a fixed fraction $f_w > 0$ of writes, the protocol's throughput is
    always upper bounded by a constant proportional to $\frac{1}{f_w}$, even
    with an infinite amount of resources.
\end{itemize}
