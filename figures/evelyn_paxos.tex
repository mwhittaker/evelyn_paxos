% Processes.
\tikzstyle{proc}=[draw, circle, thick, inner sep=2pt]
\tikzstyle{igproc}=[draw=gray, text=gray]
\tikzstyle{client}=[proc, fill=clientcolor!25]
\tikzstyle{proposer}=[proc, fill=proposercolor!15]
\tikzstyle{proxyleader}=[proc, fill=proxyleadercolor!15]
\tikzstyle{acceptor}=[proc, fill=acceptorcolor!25, font=\scriptsize]
\tikzstyle{replica}=[proc, fill=replicacolor!25]

% Process labels.
\tikzstyle{proclabel}=[inner sep=0pt, align=center, font=\footnotesize]

% Components.
\tikzstyle{component}=[draw, thick, flatgray, rounded corners]

% Messages and communication.
\tikzstyle{comm}=[-latex, thick]
\tikzstyle{commnum}=[fill=white, inner sep=0pt]

\begin{figure}[ht]
  \centering
  \begin{tikzpicture}[xscale=1.5]
    % Processes.
    \node[client] (c1) at (0, 2) {$c_1$};
    \node[client] (c2) at (0, 1) {$c_2$};
    \node[client] (c3) at (0, 0) {$c_3$};
    \node[proposer, igproc] (p1) at (1, 1.5) {$p_1$};
    \node[proposer, igproc] (p2) at (1, 0.5) {$p_2$};
    \node[proxyleader, igproc] (pl1) at (2, 2) {$l_1$};
    \node[proxyleader, igproc] (pl2) at (2, 1) {$l_2$};
    \node[proxyleader, igproc] (pl3) at (2, 0) {$l_3$};
    \node[acceptor] (a01) at (3, 2) {$a^0_1$};
    \node[acceptor] (a02) at (3.5, 2) {$a^0_2$};
    \node[acceptor] (a03) at (4, 2) {$a^0_3$};
    \node[acceptor] (a11) at (3, 0) {$a^1_1$};
    \node[acceptor] (a12) at (3.5, 0) {$a^1_2$};
    \node[acceptor] (a13) at (4, 0) {$a^1_3$};
    \node[replica] (r1) at (5, 2) {$r_1$};
    \node[replica] (r2) at (5, 1) {$r_2$};
    \node[replica] (r3) at (5, 0) {$r_3$};

    % Labels.
    \crown{(p1.north)++(0,-0.15)}{0.333}{0.25}
    \node[proclabel] (clients) at (0, 3) {Clients};
    \node[proclabel, text=gray] (proposers) at (1, 3) {$f+1$\\Proposers};
    \node[proclabel, text=gray] (proxyleaders) at (2, 3) {$\geq f+1$\\Proxy\\Leaders};
    \node[proclabel] (acceptors) at (3.5, 3) {$\geq 1$ group of \\$2f+1$ Acceptors};
    \node[proclabel] (replicas) at (5, 3) {$\geq f+1$\\Replicas};
    \halffill{clients}{clientcolor!25}
    \quarterfill{proposers}{proposercolor!25}
    \quarterfill{proxyleaders}{proxyleadercolor!25}
    \quarterfill{acceptors}{acceptorcolor!25}
    \quarterfill{replicas}{replicacolor!25}

    % Communication.
    \draw[comm, bend right=20] (c2) to node[commnum]{1} (a11);
    \draw[comm, bend left=10] (c2) to node[commnum]{1} (a12);
    \draw[comm, bend left=35] (a11) to node[commnum]{2} (c2);
    \draw[comm, bend left=5] (a12) to node[commnum]{2} (c2);
    \draw[comm, bend left=7] (c2) to node[commnum]{3} (r1);
    \draw[comm, bend left=7] (r1) to node[commnum]{4} (c2);
  \end{tikzpicture}
  \caption{%
    An example Evelyn Paxos read. Proposers and proxy leaders are grayed
    because they are not involved in reads. An Evelyn Paxos write is identical
    to a Compartmentalized MultiPaxos write; see
    \figref{MultiPaxosMoreReplicasDiagram}.
  }%
  \figlabel{EvelynPaxos}
\end{figure}
